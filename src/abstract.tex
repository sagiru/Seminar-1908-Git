\begin{abstract}
\section*{\abstrc}
\label{sec:abstract_de}
Diese Seminararbeit stellt Versionskontrollsysteme und deren Funktion im
Allgemeinen und Git im Speziellen vor. Neben einem geschichtlichen \"Uberblick
wird, nach einem kurzen Exkurs in den Bereich Kollaboration, hierbei auf
grundlegende Eigenschaften von Versionskontrollsystemen eingegangen. Ebenso
werden wesentliche Unterschiede, insbesondere gegen\"uber Git, aufgezeigt. Um
einen praktischen Einblick in die Arbeit mit Git zu vermitteln, werden die
gebr\"auchlichsten Befehle und Vorgehensweisen in Beispielen vorgestellt.
Nachdem in groben Zügen auf das Objektmodell eingegangen wurde werden die
eingesetzten Befehle nochmal in einer Kommandoübersicht ergänzt. Anschließend
finden auch einige Enschränkungen Erwähnung. Um die Leistungsfähigkeit in der
Praxis zu unterstreichen, wird zum Abschluss das Projekt vorgestellt, für das
Git ursprünglich entwickelt wurde - Der Linux Kernel.

Insgesamt soll diese Arbeit einen ersten Einblick auf den Einsatz von Git in
Teams, Projekten und Betrieben vermitteln und warum Versionskontrollsysteme
essenziell f\"ur den Einsatz in erfolgreichen Softwareprojekten sind. Mit
R\"ucksicht auf den Umfang der Arbeit, werden Details hierzu aber nicht weiter
ausgef\"uhrt.
\end{abstract}
