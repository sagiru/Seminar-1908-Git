\begin{abstract}
\section*{\abstrc}
\label{sec:abstract_de}
Diese Seminararbeit stellt Versionskontrollsysteme und deren Funktion im
Allgemeinen und Git im Speziellen vor. Neben einem geschichtlichen \"Uberblick
wird, nach einem kurzen Exkurs in den Bereich Kollaboration, hierbei auf
grundlegende Eigenschaften von Versionskontrollsystemen eingegangen. Ebenso
werden wesentliche Unterschiede aufgezeigt. Dabei wird insbesorndere auf Git
eingegangen. Um einen praktischen Einblick in die Arbeit mit Git zu vermitteln,
werden die gebr\"auchlichsten Befehle und Vorgehensweisen in Beispielen
vorgestellt.  Nachdem in groben Zügen auf das Objektmodell eingegangen wurde,
werden die eingesetzten Befehle nochmal in einer Kommandoübersicht durch
weitere Befehle ergänzt. Anschließend werden auch einige Enschränkungen von Git
beschrieben. Um die Leistungsfähigkeit in der Praxis zu unterstreichen, wird
zum Abschluss das Projekt vorgestellt, für das Git ursprünglich entwickelt
wurde - Der Linux Kernel.

Insgesamt gibt diese Arbeit einen ersten Einblick in den Einsatz von Git in
Teams, Projekten und Betrieben. Sie verdeutlicht, warum Versionskontrollsysteme
essenziell f\"ur den Einsatz in erfolgreichen Softwareprojekten sind.
\end{abstract}
