\begin{abstract}
\section*{\abstrc}
\label{sec:abstract_de}
Diese Seminararbeit stellt Versionskontrollsysteme und deren Funktion im
Allgemeinen und Git im Speziellen vor. Neben einem kurzen geschichtlichen
\"Uberblick, wird hierbei auf grundlegende Eigenschaften von
Versionskontrollsystemen eingegangen. Ebenso werden wesentliche Unterschiede,
insbesondere gegen\"uber Git, aufgezeigt. Um einen praktischen Einblick in die
Arbeit mit Git zu vermitteln, werden die gebr\"auchlichsten Befehle in
Beispielen vorgestellt. Dar\"uber hinaus wird auf eine kleine Auswahl von
Workflows und fortgeschrittener Konzepte eingegangen. Zum Abschluss wird ein
Ausblick auf den Einsatz von Git in Teams, Projekten und Betrieben gegeben.
Dieser Ausblick soll unter anderem Bezug auf die Frage nehmen, warum ein
Einsatz von Versionskontrollsystemen essenziell f\"ur den Einsatz in
erfolgreichen Softwareprojekten ist. Mit R\"ucksicht auf den Umfang der Arbeit
werden Details hierzu aber nicht weiter ausgef\"uhrt.
\end{abstract}
