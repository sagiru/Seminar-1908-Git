\newglossaryentry{Umgebung}{
  name=Umgebung,
  description={
    In der Softwareentwicklung wird häufig von produktiven Umgebungen
    gesprochen. Damit ist, aus betrieblicher Sicht, eine Zusammenstellung aus
    verschiedenen Systemen in definierter Konfiguration gemeint, auf denen die
    entwickelte Software produktiv eingesetzt wird. Um Fehlentwicklungen zu
    vermeiden, wird die Software häufig in sogenannten Entwicklungsumgebungen,
    die der Produktionsumgebung nachempfunden sind, getestet \cite[S.~49,
    250]{cd}
  }
}
\newglossaryentry{wrapper}{
  name=Wrapper,
  description={
    Als Wrapper wird eine Software bezeichnet, die eine andere Software
    vollständig oder teilweise unmschließt und ggf. um weitere Funktionalität,
    ergänzt oder verändert.
  }
}

\newglossaryentry{OpenSource}{
  name=Open Source,
  description={
    Mit Open Source(OS) Software wird Software bezeichnet, deren Quelltext frei
    und öffentlich verfügbar ist. Ausschlaggebend ist, dass er der
    Allgemeinheit uneingeschränkt zur Verfügung steht, bzw. auf Anfrage zur
    Verfügung gestellt wird. Dieser darf also gelesen, genutzt und verändert
    werden. Die Veränderungen können, dürfen und sollten der Allgemeinheit
    ebenso wieder zur Verfügung gestellt werden. I.d.R kann Open Source
    Software kostenlos genutzt werden. Dieses Vorgehen wird von der Open Source
    Gemeinschaft (Open Source Community) durch allgemein anerkannte Regeln und
    Lizenzmodelle bekräftigt\cite{osi_d,osi_l}.  }
}

\newglossaryentry{tag}{
  name=Tag,
  description={
    Siehe Abschnitt \ref{sec:tag} auf Seite \pageref{sec:tag}.
  }
}

\newglossaryentry{commit}{
  name=Commit,
  description={
    Siehe Abschnitt \ref{sec:commit} auf Seite \pageref{sec:commit}.
  }
}

\newglossaryentry{repository}{
  name=Repository,
  description={
    Siehe Abschnitt \ref{sec:repository} auf Seite \pageref{sec:repository}.
  }
}

\newglossaryentry{SHA-1}{
  name=SHA-1,
  description={
    \textit{Secure Hash Algorithm} (SHA-1) ist ein, für kryptografische Zwecke
    entwicklter, Hash-Algotithmus der eine Prüfsumme aus digitalen Daten z.B.
    Dateien bildet. Es handelt sich um eine nicht umkehrbare mathematische
    Funktion die eine Bit-Folge auf eine Prüfsumme von 160 Bit Länge
    abbildet.\cite[S.~50]{gitosp} }
}
