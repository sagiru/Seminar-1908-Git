\newglossaryentry{Umgebung}{
  name=Umgebung,
  description={
    In der Softwareentwicklung wird häufig von
    produktiven Umgebungen die gesprochen.  Damit ist eine Zusammenstellung aus
    verschiedenen Systemen, in definierter Konfiguration, gemeint auf denen die
    entwickelte Software produktiv eingesetzt wird. Um Fehlentwicklungen zu
    vermeiden, wird die Software häufig in sogenannten Entwicklungsumgebungen,
    die der Produktionsumgebung nachempfunden sind, getestet \cite[s.~49, 250]{cd}
  }
}
\newglossaryentry{OpenSource}{
  name=Open Source,
  description={Mit Open Source(OS) Software wird Software bezeichnet, deren
  Quelltext frei und öffentlich verfügbar ist. Auschlaggebend ist das er der
  Allgemeinheit uneingeschränkt zur Verfügung steht bzw. auf Anfrage zur
  Verfügung gestellt wird. Dieser darf also gelesen, genutzt und verändert
  werden. Die Veränderungen können und dürfen der Allgemeinheit ebenso wieder
  zur Verfügung gestellt werden. I.d.R kann Open Source Software kostenlos
  genutzt werden. Dieses vorgehen wird von der Open Source Gemeinschaft (Open
  Source Community) durch allgemein annerkannte Regeln und Lizenzmodelle
  bekräftigt\cite{osi_d,osi_l}.
  }
}
