\chapter{\preamble}
\label{cha:preamble}
Das heutige Umfeld der Informationstechnologie ist gepr\"agt von einer Vielzahl
unterschiedlichster Arten von Software. Schaut man sich allein in seinem Alltag
genauer um, wird man feststellen, daß man es täglich mit verschiedensten
Systemen zu tun hat, auf denen eine Softwarekomponente ihre Arbeit verrichtet.
Allen voran ist sicherlich das Internet, mit zahlreichen Online Plattformen ein
Motor dieser Softwareindustrie. Dieser o.a. Ausschnitt ist aber auch nur ein
Teil, mit dem man unmittelbar Kontakt hat oder den man umgangssprachlich
"`sehen"' kann.  Eine Vielzahl von Softwaresystemen wird hinter verschlossenen
T\"uren entwickelt und betrieben. So entwickeln z.B.  Versicherungen, Banken
o.\"a.  Organisationen viele ihrer Softwaresysteme nicht f\"ur die
\"Offenlichkeit. Eine Eigenschaft die fast alle gro\ss{}en Softwareprojekte
gemeinsam haben ist, daß sie i.d.R. in Teams unterschiedlicher Gr\"o\ss{}e
entwickelt werden. Je nach Zusammenstellung dieser Teams, kommt es hier u.U. zu
geografischen und zeitlichen Herausforderungen solche Systeme gemeinsam
effektiv zu entwickeln.  Weitere Beispiele f\"ur verteilte Softwareprojekte
findet man ebenfalls h\"aufig in der Welt der \gls{OpenSource} Software. So ist
Git\footnote{\label{git:1}\url{https://git-scm.com/}}, auf das im Weiteren noch
eingegangen wird, ebenfalls ein Beispiel aus der \gls{OpenSource} Community. So
stellen sich nach Jez Humble und David Farley in Projekten mit verteilten Teams
einige Fragen, auf die in Kapitel \ref{cha:Versionsverwaltung} Bezug genommen
wird\cite[S.~26, 33]{cd}:
\begin{itemize}
  \item Ein neues Teammitglied bekommt seinen Arbeitsplatz und ben\"otigt
  schnellen Zugriff auf ggf. verschiedene Versionen der entwickelten Software,
  m\"ochte diese ggf. kompilieren und in einem definierten Umfeld ausf\"uhren.
  Auch gibt es vielleicht verschiedene Umgebungen und der Entwickler muss
  feststellen, welche Version mit welchen Unterschieden in welcher \gls{Umgebung}
  \footnote{\label{umgebung:1}In der Softwareentwicklung wird häufig von
  produktiven Umgebungen gesprochen. Damit ist eine Zusammenstellung aus
  verschiedenen Systemen in definierter Konfiguration gemeint, auf denen die
  entwickelte Software produktiv eingesetzt wird. Um Fehlentwicklungen zu
  vermeiden, wird die Software häufig in sogenannten Entwicklungsumgebungen,
  die der Produktionsumgebung nachempfunden sind getestet \cite[S.~49,
  250]{cd}.} gerade verwendet wird \citep[S.~26]{cd}.
  \item Was macht eine spezielle Version einer Software aus?  Wie kann eine
  konkrete Auspr\"agung einer Software, die sich im produktiven Einsatz
  befindet, reproduziert werden. Das kann z.B. n\"otig sein, um einen akuten
  Fehlerfall nachzustellen\cite[s.~33]{cd}.
  \item Was wurde durch wen und wann ge\"andert? Und aus welchen Gr\"unden?
  Auch das kann f\"ur den Fall einer Fehlersuche von Vorteil sein. Darüber
  hinaus kann mit Beantwortung dieser Frage, die Projekthistorie aus Sicht der
  Entwicklung reproduziert werden\cite[S.~33]{cd}.
\end{itemize}
Die o.a. Fragen sind natürlich nicht nur exklusiv auf große Projekte
beschränkt, sondern gewinnen mit steigender Größe verstärkt an Wichtigkeit.
Vergleichbare Problemstellungen finden sich aber genauso in
Entwicklungsprojekten bei denen nur wenige Personen beteiligt sind. Um diese
Fragestellungen, neben der eigentlichen Frage wofür Versionskontrollsysteme
überhaupt eingesetzt werden, zu berücksichtigen wird in Kapitel
\ref{cha:Versionsverwaltung} ab Seite \pageref{cha:Versionsverwaltung} auf
Versionskontrollsysteme(\acrshort{vcs}) im Allgemeinen eingegangen. Hier wird,
neben einer grundlegenden Definition und einem geschichtlichen Überblick auf
Gemeinsamkeiten und Vorgehensweisen eingegangen. In Kapitel \ref{cha:git} ab
Seite \pageref{cha:git} wird dann Git vorgestellt, wobei zunächst die
Grundlagen und Architektur beschrieben werden und daran anschließend die
praktische Anwendung, Workflows und fortgeschrittener Konzepte. Um einen
weiteren Praxisbezug herzustellen sowie weitere Vorteile und Möglichkeiten von
Git herauszustellen, wird abschließend in Kapitel \ref{cha:lookout} ab Seite
\pageref{cha:lookout} der Einsatz von Git in gößeren Projekten und Unternehmen
betrachtet.
