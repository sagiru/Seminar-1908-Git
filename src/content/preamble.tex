\chapter{\preamble}
\label{cha:preamble}
Das heutige Umfeld der Informationstechnologie ist gepr\"agt von einer Vielzahl
unterschiedlicher Arten von Software. Allein das Alltagsleben stüzt sich auf
täglich auf verschiedenste Systeme, auf denen eine Softwarekomponente ihre
Arbeit verrichtet.  Allen voran ist sicherlich das Internet mit zahlreichen
Online Plattformen zu nennen.  Eine Eigenschaft, die fast alle gro\ss{}en
Softwareprojekte gemeinsam haben ist, daß sie in Teams unterschiedlicher
Gr\"o\ss{}e entwickelt werden. Je nach Zusammenstellung dieser Teams, kommt es
hier möglicherweise zu geografischen und zeitlichen Herausforderungen solche
Systeme gemeinsam effektiv zu entwickeln.  Weitere Beispiele f\"ur verteilte
Softwareprojekte findet man h\"aufig in der Welt der \gls{OpenSource} Software.
So ist Git\footnote{\label{git:1}\url{https://git-scm.com/}}, auf dem der Fokus
dieser Arbeit liegt, ebenfalls ein Beispiel aus der \gls{OpenSource} Community. 

Nach Jez Humble und David Farley, stellen sich in Projekten mit verteilten
Teams einige Herrausforderungen, auf die in Kapitel
\ref{cha:Versionsverwaltung} Bezug genommen wird \cite[S.~26, 33]{cd}:
\begin{itemize}
  \item Ein neues Teammitglied bekommt seinen Arbeitsplatz und ben\"otigt
  schnellen Zugriff auf ggf. verschiedene Versionen der entwickelten Software,
  m\"ochte diese ggf. kompilieren und in einem definierten Umfeld ausf\"uhren.
  Auch gibt es vielleicht verschiedene Umgebungen und der Entwickler muss
  feststellen, welche Version mit welchen Unterschieden in welcher \gls{Umgebung}
  \footnote{\label{umgebung:1}In der Softwareentwicklung wird häufig von
  produktiven Umgebungen gesprochen. Damit ist eine Zusammenstellung aus
  verschiedenen Systemen in definierter Konfiguration gemeint, auf denen die
  entwickelte Software produktiv eingesetzt wird. Um Fehlentwicklungen zu
  vermeiden, wird die Software häufig in sogenannten Entwicklungsumgebungen,
  die der Produktionsumgebung nachempfunden sind getestet \cite[S.~49,
  250]{cd}.} gerade verwendet wird \citep[S.~26]{cd}.
  \item Was macht eine spezielle Version einer Software aus? Wie kann eine
  konkrete Auspr\"agung einer Software, die sich im produktiven Einsatz
  befindet, reproduziert werden. Reproduzierbarkeit kann z.B. n\"otig sein, um
  einen akuten Fehlerfall nachzustellen \cite[s.~33]{cd}.
  \item Was wurde durch wen und wann ge\"andert? Und aus welchen Gr\"unden?
  Auch das kann f\"ur den Fall einer Fehlersuche von Vorteil sein. Darüber
  hinaus kann mit Beantwortung dieser Frage, die Projekthistorie aus Sicht der
  Entwicklung reproduziert werden \cite[S.~33]{cd}.
\end{itemize}
Diese Fragen sind natürlich nicht auf große Projekte beschränkt, sondern
gewinnen hier vermehrt an Wichtigkeit. Vergleichbare Problemstellungen finden
sich aber genauso in Entwicklungsprojekten, an denen nur wenige Personen
beteiligt sind. 

In Kapitel \ref{cha:Versionsverwaltung} wird vorgestellt, wie
Versionskontrollsysteme bei der Lösung dieser Problemstellungen helfen können
und was ihre eigentlichen Aufgaben sind. Hier wird, neben einer grundlegenden
Definition und einem geschichtlichen Überblick auf Gemeinsamkeiten und
Vorgehensweisen verschiedener Versionskontrollsysteme eingegangen. In Kapitel
\ref{cha:git} wird das Versionskontrollsystem Git vorgestellt. Hierbei werden
zunächst die Grundlagen beschrieben, anschließend wird eine praktische
Anwendung vorgestellt. Aufbauend darauf wird auf grundlegende Workflows und die
Arbeitsweise von Git eingegangen. Um einen weiteren Praxisbezug herzustellen
sowie weitere Vor- und Nachteile von Git herauszustellen, werden in Kapitel
\ref{cha:lookout} zuerst Einschränkungen von Git und abschließend der Einsatz
von Git im Linux Kernel Projekt betrachtet.
