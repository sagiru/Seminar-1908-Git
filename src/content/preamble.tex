\chapter{\preamble}
\label{cha:preamble}
Das heutige Umfeld der Informationstechnologie ist gepr\"agt von einer Vielzahl
unterschiedlicher Varianten von Software. Schaut man sich allein in seinem
Alltag genauer um wird man feststellen das man es tagt\"aglich mit
verschiedensten Systemen zu tun hat auf eine Softwarekomponente ihre Arbeit
verichtet. Allen voran ist sicherlich das Internet mit den zahlreichen Online
Platformen ein Motor dieser Softwareindustrie. Dieser o.a. Ausschnitt ist aber
auch nur ein Teil mit dem man unmittelbar Kontakt hat oder den man
umgansprachlich "sehen" kann.  Eine Vielzahl von Softwaresystemen wird hinter
verschlossenen T\"uren entwickelt und betrieben. So entwickeln z.B.
Versicherungen, Banken o.\"a. Organisationen ihre Softwaresysteme nicht f\"ur die
\"Offenlichkeit. Eine Eigenschaft die fast alle gro\ss{}en Softwareprojekten
gemeinsam haben ist das Sie i.d.R. in Teams unterschiedlicher Gr\"o\ss{}e entwickelt
werden. Je nach Zusammenstellung solcher Teams kommt es hier u.U. zu
geografischen und zeitlichen herausforderungen solche Systeme gemeinsam
effektiv zu entwickeln. Weitere Beispiele f\"ur verteilte Softwareprojekte findet
man ebenfalls h\"aufig in der Welt der Open Source Software. So ist Git, auf das
wir im weiteren noch eingehen werden, ebenfalls ein Beispiel aus der Open
Source Community. An dieser Stelle seien nur einige wenige solcher Herrausforderungen
aufgelistet die f\"ur das Thema der Seminararbeit gewisse relevanz haben:
\begin{itemize}
\item Ein neues Teammitglied bekommt seinen Arbeitsplatz und ben\"otigt
schnellen Zugriff auf ggf. verschiedene Versionen der entwickelten Software.
M\"ochte diese ggf. Compilieren und in einer Umgebung ausf\"uhren. Auch gibt es
vielleicht verschiedene Umgebungen und der Entwickler muss Feststellen welche
Version mit welchen Unterschieden in welcher Umgebung gerade verwendet wird.
Was genau sich hinter solch einer Umgebung verbirgt soll hier nicht weiter
ausgef\"uhrt werden.
\item Was macht eine spezielle Version einer Software aus? Wie kann eine
konkrete Auspr\"agung von einer Software, die sich im produktiven Einsatz
befinden, reproduziert werden. Das kann .z.B. n\"otig Sein um einen Fehlerfall
nachzustellen.
\item Was wurde durch wen und wann ge\"andert? Und aus welchen Gr\"unden? Auch das
kann f\"ur den Fall einer Fehlersuche von Vorteil sein. So kann, wenn diese Frage
beantwortet werden kann, auch die Historie der Entwicklung reproduziert werden.
\end{itemize}
