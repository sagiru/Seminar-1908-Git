\chapter{\result}\label{cha:result}
Im Rahmen dieser Arbeit wurde grundlegend über Funktionen und Arbeitsweisen von
Versionskontollsystemen im allgemeinen und Git im speziellen gesprochen. Die
angesprochenen Themen wurden meist nur Oberflächlich betrachtet und
fortgeschrittene Betrachtungen aus Platzgründen häufig vollständig weggelassen.
Allein aber die Tatsache das alle Dateien über die Zeit in jeder Version
zuverlässig und reproduzierbar erhalten bleiben ist in erheblicher zugewinn zu
einer Zeit wo Quellcode ein einsames dasein auf der Festplatte eines Computers
frisitete. Jez Humble und David Farley sprechen in \cite[S.~35]{cd} von:

\begin{center}
\textit{\glqq{}Version Control: The Freedom do Delete}.
\end{center}

Damit meinen die Autoren nicht das leichtere Verwalten von Dateien sondern eher
die Idee das \textit{alte} Ideen oder Entwicklungen ohne Risiko von Verlusten
entfernt oder überabreitet werden können. Diese Sicherheit schafft
\textit{Freiheit} neue Dinge auszuprobieren oder zu verbessern.

Ungeachtet der in Abschnitt \ref{cha:lookout} beschriebenen Einschränkungen ist
Git im Vergleich zu vorhergehenden Versionskontrollsystemen ein mächtiges
Werkzeug welches nicht nur technische Verbesserungen mitbringt sondern die
Zusammenarbeit in Teams und Projekten auch äußerst effizient unterstüzen kann.
Die Bedienung von Git, hat man nun Erfahrungen mit anderen
Versionskontollsystemen oder nicht, mag zu Anfang etwas sperrig sein.  Wenn man
sich aber mit fortgeschrittenen Themen im Rahmen umfangreicherer Projekten
auseinandersetzt (z.B. Linux Abschnitt \ref{sec:kernel}) stellt man fest das
die Architektur von Git eine Skalierbarkeit ermöglicht die man mit anderen
Systemen vergeblich sucht. Das in der Arbeit verwendetet Beispiel (Abschnitt
\ref{sec:first_commits}) Zeigt aber auch die Arbeit mit Git in einem kleinen Rahmen. So wurde
auch diese Semiararbeit, die in LaTeX gesetzt wurde, mit Git
Versioniert\footnote{https://github.com/sagiru/Seminar-1908-Git/}. Auch
\textit{kleine} Projekte die nicht in Teams geteilt und bearbeitet profitieren
von der Arbeit mit Git.

Die Qualität von erstelltem Quellcode und Kollaboration in Teams kann durch den
Einsatz von Platformen wie Gerrit\footnote{https://www.gerritcodereview.com/}
oder GitHub\footnote{https://github.com} nochmals verbessert werden. Der
Einsatz von Versionskontrollsystemen ist sowohl Vorrausetzung für eine
erfolgreiche Auseinandersetzung mit Themen wie \textit{Continuous Integration}
oder \textit{Continuous Delivery} als auch für den Einsatz agiler
Vorgehensmodelle in der Softwareentwicklung.\cite{cd}

Schlussendlich sei dem interessierten Leser noch ans Herz gelegt sich mit
weiterführenden Begriffen wie Regression und Bisektion, Stashes, Submodulen,
Workflows und Reviewsystemen auseinander zu setzten. Diese Themen werden in der
einschlägigen Fachliteratur\footnote{\cite{gitosp},\cite{progit},\cite{gitwf}
oder \cite{cd}} entsprechend ausführlich behandelt.

\chapter{TODOs}
\begin{itemize}
\item Fix the abstract because i reduced the content!!!!
\item Fazit should contain a result how good git is for collaboration 
\item Tools like gerrit for fazit?
\item Fazit should also give a hint to a lot of more stuff but no space to write anymore 
\item What about Regressionen und Bisektion
\item Second lookout could be a section with "Git in Unternehmen und Projekten"
or "Git im Internet" woith max a half page content
\item Several HEAD revisions in \ref{sec:head}
\item Version Control: The Freedom to Delete cd S. 35
\item No binaries in VCS S.35 top
\end{itemize}
