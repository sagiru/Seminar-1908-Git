\chapter{Versionsverwaltung}\label{cha:Versionsverwaltung}
\section{Definition}\label{sec:Definition}
Versionskontrollsysteme sind auch bekannt als Versionsverwaltungssysteme (engl.
\acrlong{vcs}), Quellcode Verwaltung(engl. Source Control) oder
Revisionskontollsysteme(engl. Revision Control System). Mit diesen Begriffen
sind Systeme gemeint die es Entwicklern, Teams oder Organisationen erlauben
eine vollständige Historie mit allen Änderungen an dem Quellcode ihrer
gemeinsam entwickelten Software zu verwalten. Ausschlaggebend ist hierbei das
für alle Nutzer transparent wird wer, wann und vor allem warum welche
Änderungen durchgeführt hat. Eine weiterer wichtige Eigentschaft ist das es
verschiedenen Teams eine Zusammenarbeit an ggf\. verschiedenen Teilen der
Software ermöglicht ohne sich gegenseitig zu
behindern\footnote{\label{dev:1}Das hängt natürlich nicht nur von dem
Versionskontollsystem ab sondern auch von dem Design der entwickelten Software.
Diese wird i.d.R. eher modular aufgebaut so das die Möglichkeit einer paralelle
Entwicklung unterstützt wird.}.\cite[s.~381]{cd}

\section{Geschichtliche Entwicklung}\label{sec:GeschichtlicheEntwicklung}
Das erste Versionskontrollsystem namens SCCS enstand 1972 und wurde von Marc J.
Rockkind bei Bell Labs
geschrieben\footnote{\url{http://www.belllabs.com/}}\cite[s.~382]. Ab diesem
Zeitpunkt enstand eine Vielzahl von verschiedenen Versionskontollsystemen.  Als
Alternative zu dem properitären \acrshort{sccs} folgte Anfang 1980 das von
Walter F. Tichy an Purdue University entwickelte erste \acrlong{OpenSource}
Versionskontrollsystem \acrfull{rcs}\cite{paper:rcs,link:rcs}. Ross Ridge
veröffentlichte 1993 mit einer Beta Version von \acrshort{mysc} einen freien
Ersatz für \acrshort{sccs}. In späteren Versionen wurde \acrshort{mysc} in
\acrfull{cssc} umbenannt\cite{link:cssc,link:mysc}. Alle drei Systeme finden in
der Praxis nur noch wenig Anwendung und daher wird an dieser Stelle nicht auf
weiter auf Details eingegangen. In der weiteren Auswahl sei hier nur noch auf
einige populäre Systeme eingegangen.

\subsection{CVS}\label{sec:cvs}
Das 1986 durch Dick Grune veröffentlichte \acrfull{cvs} war das erste freie
Versionskontrollsystem mit einem zentralen \gls{repository}. Das wurde ereicht
in dem \acrshort{rcs}, mit Hilfe eines \gls{wrapper}, um eine
Client-/Serverkomponente erweitert wurde. Das ermöglichte erstmals das mehrere
Entwickler gleichzeitig an einem \gls{repository} und konkurrierend an den
selben Dateien arbeiten konnten. Neben dem innovativen Ansatz gabe es hier aber
noch einige technische Einschränkungen die ein kollaboratives Arbeiten
erschwerten. So war z.B. die Nutzung des verbrauchten Speicherplatzes nicht
optimal. Das erzeugen von Abzweigungen (Branches) wurde durch einfaches
Kopieren erreicht. Das war deshalb nicht nur Zeitaufwändig sondern verbrauchte
auch entsprechenden Speicherplatz. Ein späteres Zusammenführen (mergen) dieser
Zweige führte daher zu Dateikonflikten und verursachte hierduch erheblichen
Aufwand. Auch gab es keine Funktonalität Binärdateien zu verwalten so das hier
der Speicherplatz auch eher ineffizient genutzt wurde. Das erstellen von Tags
war mit wachsendem Inhalt des \glspl{repository} ebenfalls Zeitaufwändig da
alle enthaltenen Dateien bearbeitet werden mussten. Eine der, aus heutiger
Sicht, größte Einschränkung war aber sicher die Tatsache das Commits in das
\gls{repository} nicht Atomar waren. Wurde die Übertragung der Dateien in das
zentrale \gls{repository} unterbrochen so wurde dieses in einem inkonsistenten
und nicht nutzbaren Zustand hinterlassen und musste administrativ repariert
werden. \cite[s.~382-383]{cd}

\subsection{SVN}
Das Ziel des 
\subsection{Git}
\subsection{Weitere}
\section{Versionsverwaltungssysteme}
\subsection{Lokal}
Beide Systeme arbeiteten auf dem lokalen
Dateisystem.,
\subsection{Zentral}
\subsection{Verteilte Versionskontrolle}
\subsection{Streaming}
\section{Warum und Wozu?}
single vs big teams, picture lokal vs team
\label{sec:why}
\section{Konzepte}
\label{sec:systems}
\section{Konzepte}
\label{sec:Konzepte}
\subsection{Datenhaltung}
\label{sec:Datenhaltung}
\subsection{Tags}
\label{sec:Tags}
\subsection{Branches}
\label{sec:Branches}
