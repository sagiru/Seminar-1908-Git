\chapter{Git}\label{cha:Git}
In diesem Kapitel wird auf die wesentlichsten Grundlagen und Befehle
beschrieben um mit Git Dateien zu Verwalten. Es wird als erstes mit einfachen
Befehlen ein Git-Repository erstellt und später am Beispiel dieses Repositorys
auf weitere Eigenschaften und fortgeschrittene Befehle eingegangen. Darüber
hinaus stellt der Author dieser Seminararbeit den Quellcode dieser Arbeit als
Git-Repository unter \cite{link:seminararbeit} zur Verfügung. Diese Repository
dient ebenso als Beispiel für Sachverhalte bei denen der gewünschte
Zusammenhang hiermit besser darzustellen ist.

\section{Grundlagen}\label{gitbasics}
In diesem Beispiel werden wir ein Repository \texttt{git-example} mit einem
Skript (\texttt{git-stats}) erstellen welches ein paar einfache Statistiken
über die Autoren eines git Repositorys erzeugt.

Alle Beispiele werden auf einer Linux Kommandozeile durchgeführt. Git ist für
alle gängigen Linux derivate verfügbar. Die Autoren von \cite[S.~12-14]{gitpro}
gehen auf weitere Detauls zur Installation von Git in anderen Versionen oder
auf anderen Betriebsystemen ein. 

Die in den Beispielen eingesetzte Version von ist 2.15.0.

\lstinputlisting[
  label=lst:gitversion,
  caption={Version von Git ausgeben},
  captionpos=none,
]{listings/git_version.lst}

\subsection{Konfiguration}\label{gitconfig}
Zu Beginn sollten zu erst einige grundlegende Konfigurationen vorgenommen
werden. So sollten, damit die Nachvollziehbarkeit gewährleistet ist, als erstes
immer der Name und Mailadresse des Benutzers konfiguriert werden. Auch ist die
farbliche Darstellung der Ausgaben durchaus hilfreich. Hier kann Git so
konfiguriert werden das die Farbausgabe automatisch unterdrückt wird, sollte
die Ausgabe in eine Datei umgeleitet werden.

\lstinputlisting[
  label=lst:gitinitconfig,
  caption={Erste Git Konfiguration},
  captionpos=none,
]{listings/git_init_config.lst}

Die eingestellten Optionen können ebenso direkt in der Datei
\texttt{\textasciitilde/.gitconfig} eingesehen und bearbeitet werden.

\subsection{Erstellen eines Repositorys}\label{startup}
Damit Dateien nun mit Git Versioniert werden können muss wie folgt ein ein (lokales)
\gls{repository} erstellt werden: 

\lstinputlisting[
  label=lst:gitinit,
  caption={Repository anlegen},
  captionpos=none,
]{listings/git_init.lst}

Sollte das Verzeichnis \texttt{hello\_seminar.git} noch nicht existieren wird es
angelegt. Zusätzlich wird innerhalb von \texttt{hello\_seminar.git} noch ein
weiteres Verzeichnis \texttt{.git} angelegt in dem alle Daten ablegt werden die
Git zur weiteren Verwaltung des \glspl{repository} benötigt.

Um den Status des aktuell erzeugten \glspl{repository} auszugeben können wir
den Befehl \texttt{git status} verwenden:

\lstinputlisting[
  label=lst:gitstatus,
  caption={Status des Repositorys ausgeben},
  captionpos=none,
]{listings/git_status.lst}

Die von Git erzeugte Ausgabe macht darauf Aufmerksam das noch keine Commits
erzeugt wurde und das man nun neue Dateien erstellen und hinzufügen kann. 

\lstinputlisting[
  label=lst:gitstatus,
  caption={Status des Repositorys ausgeben},
  captionpos=none,
]{listings/git_status.lst}


\subsection{Die ersten Commits}\label{first_commits}

Wir werden nun die ersten Dateien zu dem Repository hinzufügen. Dazu werden wir
als erstes zwei Dateien herunterladen. Wer diese Dateien nicht herunter laden
möchte kann auch Dateien mit gleichem Namen erstellen.

\lstinputlisting[
  label=lst:downloads,
  caption={Download der Beispieldateien},
  captionpos=none,
]{listings/downloads.lst}

\section{Architektur und Bäume}\label{sec:trees}
\label{sec:Konfiguration}
\subsection{Kommandos}
\label{sec:Kommandos}
\section{Praktische Anwendung}
\label{sec:Praxis}
\section{Workflows}
\label{sec:Workflows}
\section{Fortgeschrittene Konzepte}
\label{sec:FortgeschritteneKonzepte}
\subsection{Regressionen und Bisektion}
\label{sec:Regressionen}
\subsection{Rebase}
\label{sec:Rebase}
\subsection{Andere Versionsverwaltungssysteme und Git}
\label{sec:AndereVersionsverwaltungssystemeundGit}
