\chapter{Git}\label{cha:Git}
In diesem Kapitel wird auf die wesentlichsten Grundlagen und Befehle
beschrieben um mit Git Dateien zu Verwalten. Es wird als erstes mit einfachen
Befehlen ein Git-Repository erstellt und später am Beispiel dieses Repositorys
auf weitere Eigenschaften und fortgeschrittene Befehle eingegangen. Darüber
hinaus stellt der Author dieser Seminararbeit den Quellcode dieser Arbeit als
Git-Repository unter \cite{link:seminararbeit} zur Verfügung. Diese Repository
dient ebenso als Beispiel für Sachverhalte bei denen der gewünschte
Zusammenhang hiermit besser darzustellen ist. 

\section{Grundlagen}\label{gitbasics}

Alle Beispiele werden auf einer Linux Kommandozeile durchgeführt. Die
eingesetzte Git-Version ist 2.15.0.

\lstinputlisting[
  label=lst:gitversion,
  caption={Git Version anzeigen},
  captionpos=none,
]{listings/git_version.lst}

\subsection{Konfiguration}\label{gitconfig}
Um Dateien mit git zu Verwalten sollten zu erst einige Konfigurationen
vorgenommen werden. So sollten, damit die Nachvollziehbarkeit gewährleistet
ist, als erstes immer der Name und Mailadresse des Benutzers konfiguriert
werden. Auch ist die farbliche Darstellung der Ausgaben durchaus hilfreich.
Hier kann Git so konfiguriert werden das es die Farbausgabe automatisch
unterdrückt sollte die Ausgabe in Dateien umgeleitet werden.

\lstinputlisting[
  label=lst:gitinitconfig,
  caption={Initiale Git Konfiguration},
  captionpos=none,
]{listings/git_init_config.lst}

\subsection{Erstellen eines Repositorys}\label{startup}
\lstinputlisting[
  label=lst:gitinit,
  caption={Git Repository erstellen},
  captionpos=none,
]{listings/git_init.lst}

\section{Architektur und Bäume}\label{sec:trees}
\label{sec:Konfiguration}
\subsection{Kommandos}
\label{sec:Kommandos}
\section{Praktische Anwendung}
\label{sec:Praxis}
\section{Workflows}
\label{sec:Workflows}
\section{Fortgeschrittene Konzepte}
\label{sec:FortgeschritteneKonzepte}
\subsection{Regressionen und Bisektion}
\label{sec:Regressionen}
\subsection{Rebase}
\label{sec:Rebase}
\subsection{Andere Versionsverwaltungssysteme und Git}
\label{sec:AndereVersionsverwaltungssystemeundGit}
