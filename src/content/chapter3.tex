\chapter{Git}\label{cha:Git}
Git ist ein verteiltes bzw. dezentrales \acrlong{vcs}. Es wurde von Linux
Torvalds und Junio Hamano entwickelt und ist für die gängigen Platformen wie
Linux, BSD und Windows u.a. verfügbar. Der Name Git kommt nach Torvalds wie
folgt zustande: \`I'm an egotistical bastard, and I name all my projects after
myself. First 'Linux', now 'Git'\'. Alternativ stellt Linux Torvalds noch
folgende Varianten als Acronym für Git zur Verfügung\cite{link:gitfaq}:

\begin{itemize}
  \item \glqq{} Random three-letter combination that is pronounceable, and not
  actually used by any common UNIX command. The fact that it is a
  mispronunciation of "get" may or may not be relevant.\grqq{}
  \item \glqq{} Stupid. Contemptible and despicable. Simple. Take your pick from the
  dictionary of slang.\grqq{}
  \item \glqq{} Global information tracker": you're in a good mood, and it actually
  works for you. Angels sing and light suddenly fills the room.\grqq{}
  \item \glqq{} Goddamn idiotic truckload of sh*t": when it breaks\grqq{}
\end{itemize}
\section{Architektur}
\label{sec:Architektur}
\section{Grundlagen}
\label{sec:Grundlagen}


\url{https://git.wiki.kernel.org/index.php/GitFaq#Why_the_.27Git.27_name.3F}
\subsection{Konfiguration}
\label{sec:Konfiguration}
\subsection{Kommandos}
\label{sec:Kommandos}
\section{Praktische Anwendung}
\label{sec:Praxis}
\section{Workflows}
\label{sec:Workflows}
\section{Fortgeschrittene Konzepte}
\label{sec:FortgeschritteneKonzepte}
\subsection{Regressionen und Bisektion}
\label{sec:Regressionen}
\subsection{Rebase}
\label{sec:Rebase}
\subsection{Andere Versionsverwaltungssysteme und Git}
\label{sec:AndereVersionsverwaltungssystemeundGit}
